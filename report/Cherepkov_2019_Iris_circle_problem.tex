\documentclass[12pt,twoside]{article}
\usepackage{jmlda}

%%
\newcommand{\IrisRadius}{r_{\text{iris}}}
\newcommand{\IrisRadiusTrue}{\widetilde{r}_{\text{iris}}}
%
\newcommand{\IrisCoordI}{i_{\text{iris}}}
\newcommand{\IrisCoordITrue}{\widetilde{i}_{\text{iris}}}
%
\newcommand{\IrisCoordJ}{j_{\text{iris}}}
\newcommand{\IrisCoordJTrue}{\widetilde{j}_{\text{iris}}}
%%
\newcommand{\PupilRadius}{r_{\text{pupil}}}
\newcommand{\PupilRadiusTrue}{\widetilde{r}_{\text{pupil}}}
%
\newcommand{\PupilCoordI}{i_{\text{pupil}}}
\newcommand{\PupilCoordITrue}{\widetilde{i}_{\text{pupil}}}
%
\newcommand{\PupilCoordJ}{j_{\text{pupil}}}
\newcommand{\PupilCoordJTrue}{\widetilde{j}_{\text{pupil}}}



%\NOREVIEWERNOTES
\title
    [Алгоритм аппроксимации границ радужки глаза] % Краткое название; не нужно, если полное название влезает в~колонтитул
    {Быстрый беспереборный алгоритм аппроксимации границ радужки глаза}
\author
    [Черепков~А.\,Ю., Нейчев~Р.\,Г.] % список авторов для колонтитула; не нужен, если основной список влезает в колонтитул
    {Черепков~А.\,Ю.$^1$, Нейчев~Р.\,Г.$^1$, Матвеев~И.\,А., Стрижов~В.\,В.$^2$} % основной список авторов, выводимый в оглавление
\thanks
    {Научный руководитель:  Стрижов~В.\,В.
   Задачу поставил:  Матвеев~И.\,А
    Консультант:  Нейчев~Р.\,Г.}
\email
    {cherepkov.ayu@phystech.edu}
\organization
    {$^1$Московский физико-технический институт, Россия, г. Долгопрудный, Институтский пер., 9 $^2$Вычислительный центр им. Дородницына ФИЦ «Информатика и управление» РАН}
\abstract
    {В статье рассматривается задача выделения внутренней и внешней границ зрачка на изображении глаза. Границы аппроксимируются окружностями с неизвестными параметрами: центр и радиус. Для решения поставленной задачи используются следующие шаги: морфологическая обработка с последующей бинаризацией изображения глаза; преобразование координат, переводящее окружности в прямые; поиск параметров аппроксимирующих окружностей с помощью решения задачи мультимоделирования в новых координатах. Проверка качества алгоритма производится на изображениях глаз из открытых наборов данных \cite{dataset}.

\bigskip
\textbf{Ключевые слова}: \emph {компьютерное зрение, мультимоделирование, преобразование пространства, аппроксимирующие окружности, границы радужки глаза}.}
\titleEng
    {JMLDA paper example: file jmlda-example.tex}
\authorEng
    {Author~F.\,S.$^1$, CoAuthor~F.\,S.$^2$, Name~F.\,S.$^2$}
\organizationEng
    {$^1$Organization; $^2$Organization}
\abstractEng
    {This document is an example of paper prepared with \LaTeXe\
    typesetting system and style file \texttt{jmlda.sty}.

    \bigskip
    \textbf{Keywords}: \emph{keyword, keyword, more keywords}.}
\begin{document}
\maketitle
%\linenumbers
\section{Введение}
Рассматривается задача выделения внутренней и внешней границ зрачка на изображении глаза.  Алгоритм аппроксимирует границы с помощью двух окружностей. Поиск параметров окружностей затрудняют следующие факторы: наличие бликов и шумов, низкое качество исходного изображения, допустимость различных ракурсов и перспектив при фотографировании. Данная задача может возникать в системах биометрии и в медицине, например, при создании аппаратов, производящих лазерную коррекцию зрения.

Ранее были представлены различные алгоритмы для решения поставленной задачи. Работа \cite{Efimov2015} основана на обобщённом преобразовании Хафа -- численного метода, позволяющего находить на изображении объекты, принадлежащие заданному классу фигур (в данном случае, классом являются окружности). Данный метод использует процедуру голосования в пространстве параметров для нахождения локальных максимумов. В работе \cite{Efimov2015} был сделан упор на сокращение перебора граничных точек при определении центра методом Хафа для ускорения алгоритма. В работе \cite{Chigrinskiy2016} описан более быстрый и точный алгоритм, основанный на использовании оператора Кэнни.

В данной статье представлен беспереборный алгоритм аппроксимации границ радужки глаза. Вводится преобразование пространства, в котором искомые окружности имеют вид прямых. Производится морфологическая обработка с последующей бинаризацией изображения глаза. Вводится предположение о том, что точки принадлежат к трём возможным классам. Два класса описывают окружности, третий -- все остальные. Также предполагается, что элементы из третьего класса сгенерированы из нормального распределения. Тогда задача поиска аппроксимирующих окружностей сводится к решению задачи линейного мультимоделирования в новых координатах.

\section{Постановка задачи}
\paragraph{Входные данные}
На вход подаётся растровое монохромное изображение $I_0$ размера $H \times W$ пикселей. Типичный размер -- $480 \times 640$ пикселей, однако возможны и другие размеры. Входное изображение представлено в чёрно-белом формате, то есть каждый пиксель определяет градацию серого цвета по шкале от 0 до 255: $\forall i=\overline{1 \ldots H},  j = \overline{1 \ldots W}: 0\leqslant I_0[i][j] \leqslant 255$.
\paragraph{Выходные данные}
На выход алгоритм выдаёт $\{\IrisCoordI, \IrisCoordJ, \IrisRadius \}$ и $\{\PupilCoordI, \PupilCoordJ, \PupilRadius \}$ -- параметры двух окружностей, аппроксимирующих границы радужки и зрачка соответственно.
\paragraph{Качество решения}
Обозначим за $\{\IrisCoordITrue, \IrisCoordJTrue, \IrisRadiusTrue \}$ и $\{\PupilCoordITrue, \PupilCoordJTrue, \PupilRadiusTrue \}$ параметры аппроксимирующих окружностей, предоставленных экспертом. Будем считать абсолютную ошибку $\Delta$, как максимум модулей отклонений предсказанных значений от истинных значений, предоставленных экспертом:
$$
\Delta := \max\{
|\IrisCoordITrue - \IrisCoordI|,
|\IrisCoordJTrue - \IrisCoordJ|,
|\IrisRadiusTrue - \IrisRadius|,
|\PupilCoordITrue - \PupilCoordI|,
|\PupilCoordJTrue - \PupilCoordJ|,
|\PupilRadiusTrue - \PupilRadius|,
\}$$
Относительную ошибку $\varepsilon$ определим как отношение абсолютной к радиусу радужной оболочки:
$$
\varepsilon := \frac{\Delta}{\IrisRadiusTrue}
$$
На зафиксированном входе алгоритм считается успешным, если относительная ошибка $\varepsilon$ меньше заранее заданного числа $\delta$, называемого точностью.
Качеством алгоритма будем называть долю изображений глаза, на которых алгоритм отработал успешно.


\bibliographystyle{unsrt}
\bibliography{Cherepkov_2019_Iris_circle_problem}
%\begin{thebibliography}{1}

%\bibitem{author09anyscience}
%    \BibAuthor{Author\;N.}
%    \BibTitle{Paper title}~//
%    \BibJournal{10-th Int'l. Conf. on Anyscience}, 2009.  Vol.\,11, No.\,1.  Pp.\,111--122.
%\bibitem{myHandbook}
%    \BibAuthor{Автор\;И.\,О.}
%    Название книги.
%    Город: Издательство, 2009. 314~с.
%\bibitem{author09first-word-of-the-title}
%    \BibAuthor{Автор\;И.\,О.}
%    \BibTitle{Название статьи}~//
%    \BibJournal{Название конференции или сборника},
%    Город:~Изд-во, 2009.  С.\,5--6.
%\bibitem{author-and-co2007}
%    \BibAuthor{Автор\;И.\,О., Соавтор\;И.\,О.}
%    \BibTitle{Название статьи}~//
%    \BibJournal{Название журнала}. 2007. Т.\,38, \No\,5. С.\,54--62.
%\bibitem{bibUsefulUrl}
%    \BibUrl{www.site.ru}~---
%    Название сайта.  2007.
%\bibitem{voron06latex}
%    \BibAuthor{Воронцов~К.\,В.}
%    \LaTeXe\ в~примерах.
%    2006.
%    \BibUrl{http://www.ccas.ru/voron/latex.html}.
%\bibitem{Lvovsky03}
%    \BibAuthor{Львовский~С.\,М.} Набор и вёрстка в пакете~\LaTeX.
%    3-е издание.
%    Москва:~МЦHМО, 2003.  448~с.
%\end{thebibliography}

% Решение Программного Комитета:
%\ACCEPTNOTE
%\AMENDNOTE
%\REJECTNOTE
\end{document}
